\setcounter{chapter}{4}
\chapter{Teoria delle Perturbazioni}

Fino a questo a momento abbiamo visto che per risolvere l'equazione di Schr\"odinger per un determinato sistema, \`e sufficiente determinare gli autovalori associati all'operatore Hamiltoniano. In particolare nei capitoli precedenti si \`e dimostrato come nel caso dell'atomo d'idrogeno e dell'oscillatore armonico sia possibile ottenere delle soluzioni analitiche esatte. Nella realt\`a non sempre si riesce a definire una soluzione esplicita del problema, per questo motivo si \`e ricercato dei \textit{metodi di approssimazione} che ci permettano di ottenere delle soluzioni analitiche approssimate del sistema di partenza in alcuni casi.

Il primo caso che andiamo a discutere \`e in riferimento alla perturbazione di una Hamiltoniana non esplicitamente dipendente dal tempo.


\section{Teoria delle perturbazioni indipendenti dal tempo}






